\section{Overview of Epilepsy}
Epilepsy is a disorder of the brain characterized by a lasting predisposition to generate spontaneous epileptic seizures, and has a numerous neurobiological, cognitive, and psychosocial consequences \cite{defEpilepsy}.
Epilepsy affects over 50 million people worldwide, making it one of the most common neurological diseases globally \cite{WHO}. Over 75\% of those with active epilepsy are untreated \cite{defeating_epilepsy}.

Epilepsy incidence is bimodally distributed with two peaks: the first in the pediatric population less than 5 years old, and the second in people over the age of 50 years. The incidence is higher in low-income countries than high-income countries, thanks a contribution of poor hygiene, poor basic sanitation and higher risk of infection \cite{THIJS2019689}.
Regardless the geographical location, the prevalence of active epilepsy is usually between 4 and 12 per 1000, with a risk factor that varies with age \cite{Fiest296}.

The risk of death for a person with epilepsy is increased compared with the risk for the general population. Mortality in epilepsy can be divided into direct (eg, status epilepticus, injuries, SUDEP \cite{Langan211}) or indirect (eg, suicide, drowning) disease-related death \cite{Devinsky779}. 

\noindent SUDEP (Sudden Unexpected Death in Epilepsy) is one of the causes of epilepsy-related death, it refers to a death in a patient with epilepsy that is not due to trauma, drowning, status epilepticus, or other known causes but for which there is often evidence of an associated seizure \cite{Nashef1997}. The rate of SUDEP increases with the duration and severity of epilepsy, and most cases happen during or right after a seizure \cite{Devinsky2011}.

Epilepsy rarely stands alone and the presence of comorbidities is the norm: more than 50\% of people with epilepsy have one or several additional medical problems \cite{THIJS2019689}. These comorbidities not only include include psychiatric conditions (e.g. depression, anxiety disorder, psychosis, autism spectrum disorder, dementia), but even somatic conditions (e.g. type 1 diabetes, arthritis, digestive tract ulcers) \cite{yuen2018epilepsy}.

  \subsection*{Definitions}
    \subsubsection*{Epilepsy}
    The given definition of Epilepsy is usually practically applied as having two unprovoked seizures occurring more than 24h apart. But the International League Against Epilepsy (ILAE) proposed that epilepsy be considered to be a disease of the brain by any of the following conditions: \cite{defEpilepsy}

    \begin{itemize}
      \item At least two unprovoked seizures occurring more than 24h apart;
      \item A single unprovoked seizure if recurrence risk is high (>60\% over the next 10 years);
      \item Diagnosis of an epilepsy syndrome.
    \end{itemize}
    \subsubsection*{Seizure}
    An epileptic seizure is the clinical manifestation of an abnormal, excessive, purposeless and synchronized electrical discharge in the neurons, that leads to brief episodes of involuntary movement that may involve a part of the body (partial) or the entire body (generalized) and are sometimes accompanied by loss of consciousness and control of bowel or bladder function. \cite{WHO}

  \subsection*{Pathophysiology}
  A seizure can be conceptualized as occurring when there is a distortion of the normal balance between excitation and inhibition within a neural network \cite{pathophysiology}. 
  
  In focal epilepsies, focal functional disruption results in seizures beginning in a localized fashion in one hemisphere,
  commonly limbic or neocortical, which then spread by recruitment of other brain areas. The site of the focus and the speed and extent of spread determine the clinical manifestation of the seizure \cite{DUNCAN2006, classification}. For generalized epilepsies, epileptogenic networks are widely distributed, involving thalamocortical structures bilaterally \cite{classification}.

  The imbalance between excitation and inhibition resulting in epileptogenic networks is not necessarily only an increase of excitation or a loss of inhibition; an aberrant increase in inhibition can also be pro-epileptogenic in some circumstances, such as absence seizures \cite{pinault2005cellular} or limbic epilepsies in the immature brain. \cite{galanopoulou2008gabaa}

  \subsection*{Types of epilepsy}
  Classification is made at three levels: seizure type, epilepsy type, and epilepsy syndromes. \cite{classification}
    \subsubsection*{Seizure type}
    Seizures are first classified by onset as either focal, generalized or unknown \ref{fig:Classification of epileptic seizures}.
    \begin{itemize}
      \item Focal Onset: Usually limited to a specific region of the brain, called the focus. Level of awareness subdivides focal seizure in those with retained awareness and impaired awareness. Retained awareness means that the person is aware of self and environment during the seizure, even if immobile. In addition, focal seizures are sub-grouped as those with motor and non-motor manifestation.
      \item Generalized Onset: Affects most or all of the brain. Typically congenital and occurs simultaneously in both hemispheres of the brain. They are almost always accompanied by impaired awareness. Generalized seizures are divided into motor and non-motor (absence) seizures.
      \item Unknown: It is the case in which the onset is missed or obscured. 
    \end{itemize}

    \begin{figure}[h]
      \centering
      \includegraphics[width=0.8\textwidth]{images/seizureTypes.png}
      \caption{The International League Against Epilepsy.\cite{Scheffer2017}}
      \label{fig:Classification of epileptic seizures}
    \end{figure}
  
    \subsubsection*{Epilepsy type}
    Epilepsies are divided into: focal, generalized, combined generalized and focal, and unknown \ref{fig:Classification of epilepsies}. The category combined epilepsy is used for those presenting both seizure types.

    \begin{figure}[h]
      \centering
      \includegraphics[width=0.8\textwidth]{images/epilepsyTypes.png}
      \caption{The International League Against Epilepsy.\cite{THIJS2019689, classification}}
      \label{fig:Classification of epilepsies}
    \end{figure}
  
  \subsection*{Causes of epilepsy}
  Epilepsy can have both genetic and acquired causes, with the interaction of these factors in many cases. Established acquired causes include serious brain trauma, stroke, tumours, and brain problems resulting from a previous infection.

  \subsection*{Treatments}
  For most of the people with epilepsy Antiepileptic Drug(AEDs) constitute the first line treatment. However, it has been reported that AEDs are effective in only 60-70\% of individuals, a percentage that is further reduced in low-income countries. \cite{DUNCAN2006}.

  Up to a third of all individuals with epilepsy are refractory to AEDs \cite{SpencerHuh2008}. Drug-resistant epilepsy is assumed after the "failure of adequate trials of two tolerated, appropriately chosen and used at correct dosage antiseizure drug schedules to achieve sustained seizure freedom" \cite{drug_resist}. In those cases alternative non-pharmacological treatments including surgery and neurostimulatory interventions should be considered.
  When surgery is not possible because of the presence of multifocal or generalized epilepsy or whenever the epileptogenic focus lies in eloquent cortex that cannot removed, neurostimulation techniques are palliative options \cite{Englot2013}.

  Three neurostimulation devices are approved by the Food and Drug Administration (FDA) for the treatment of drug resistant epilepsy \cite{kiriakopoulos_cascino_britton_2018, wong2019comparison}.
  \begin{itemize}
    \item VNS is approved for treatment of epilepsy when surgery is not possible or does not work. It is not a brain surgery and it is placed under skin on the top left section of the chest and uses wire coils to connect to the left Vagus nerve, located in your neck. It send intermittent electrical continuously. The response rate after a year is 55\%.
    \item RNS is a device that can record seizure activity directly from the brain and delivers stimulation to stop seizures. RNS is implanted near the seizure focus and is flush with the skull, it connects to electrodes that are placed inside your brain. It delivers pulses only when detects abnormal activity in the seizure focus. RNS is not compatible with RNS. The response rate after a year is 45\% and grows till 60\% in long-term.
    \item DBS sends signals to the brain electrode to stop signals that trigger a seizure. The connected DBS electrodes are typically placed inside the thalamus, and the electrical pulses are delivered constantly or not. The response rate after a year is 45\% and rise till 74\% in long-term. Side effects include depression and memory.
  \end{itemize}

  \begin{figure}[h]
    \centering
    \includegraphics[width=0.8\textwidth]{images/WongMani2019.jpeg}
    \caption{The following shows a general algorithm for efficacious device selection based on mechanism of action and focality. Other considerations (side effects, device features, and patient preferences) also drive device selection but are not captured in this flowchart. \cite{wong2019comparison}}
    \label{fig:Decision algorithm}
  \end{figure}

\section{Vagus Nerve Stimulation}
Vagus Nerve Stimulation (VNS) showed positive effects in multiple other medical conditions, including essential tremors gastroparesis \cite{KRAHL2004135}, chronic tinnitus, stroke, post-traumatic stress disorder \cite{HAYS2013275}, chronic pain, Parkinson's disease, eating disorders, multiple sclerosis, migraine and Alzheimer's disease \cite{BRONCEL202037, BeekwilderBeems2010}.

VNS was implanted first time in four epilepsy patients by Penry and Dean in 1988 \cite{Penry1990}. After several large clinical studies, it was approved for seizures by the European Community in 1994 and FDA in 1997. Clinical trials demonstrate that 24 to 48 months after the implantation of the device, 60\% of patients were considered as responders and 8\% of implanted patients became seizure free \cite{Englot2016}. Responders to VNS will be defined as those who experience 50\% or greater reduction in seizure frequency after VNS \cite{IBRAHIM2017634}. Although VNS is used in clinical practice the exact mechanistic of its effect in modulating seizures remain poorly understood.

VNS consists of a device implanted in the upper left thoracic region with a helical electrode placed around the left cervical nerve, which delivers intermittent electrical impulses to activate the vagus nerve \ref{fig:Vagus Nerve Stimulation}.
Studies in the dog show that right-sided VNS results in a greater degree of bradycardia as compared to the left-sided VNS, because right vagus nerve innervates more densely in the heart \cite{ardell1986selective}. Because on those studies VNS is indicated for use only in stimulating the left vagus nerve.

Side effects of VNS are commonly limited to coughing and/or hoarseness of the voice. In a study, voice alternation was reported in 66\% of patients on high stimulation and in 30\% on low stimulation and cough was reported in 45\% of patients. \cite{ben2001vagus} To avoid cardiac side effects, a cuff electrode in most cases is implanted on the left vagal nerve.

\begin{figure}[h]
  \centering
  \includegraphics[width=0.8\textwidth]{images/vagus_nerve_stimulation.jpg}
  \caption{Components of the VNS system.}
  \label{fig:Vagus Nerve Stimulation}
\end{figure}

  \subsection*{Vagus Nerve Anatomy and Connections}
  The vagal nerve (VN) is the longest cranial nerve and exerts a wide range of effects on the body. It comprises two nerves, the left and right vagus nerves and comprises both sensory and motor fibers.
  The vagal nerve is a mixed nerve made up of 75\% sensory(afferent) fibers responsible for the side effects observed (e.g. coughing, difficulties od swallowing, voice modification effects), and 25\% efferent fibers which mainly send feedback from heart, lungs, stomach and upper bowel.  \cite{BonazSinniger2017}.

  The majority of vagus nerve fibers are comprised of afferents and project to the nucleus tractus solitarius (NTS), which in turn sends fibers to other brainstem nuclei important in modulating the activity of subcortical and cortical circuitry. This vagus afferent network (VagAN) is
  thought to be the neural substrate of VNS efficacy \cite{HanchemWongIbrahim2018}. 
  The NTS receives direct inputs from the VN and projects to others brainstem nuclei: the locus coeruleus (LC), dorsal raphe nucleus (DRN), and parabranchial nucleus (PBN) \cite{RICARDO19781}. The functional importance of NTS connectivity in modulating seizure activity is further borne out by findings in rats that increased inhibitory gamma-aminobutyric acid (GABA) signaling or decreased excitatory glutamate signaling within the NST, reduces the susceptibility to chemically induced limbic motor seizures \cite{GABA}.

  \begin{figure}[h]
    \begin{minipage}[c]{0.5\textwidth}
      \includegraphics[width=\textwidth]{images/vagus_afferent_network.jpg}
    \end{minipage}\hfill
    \begin{minipage}[c]{0.5\textwidth}
      \caption{The vagus afferent network. Schematic diagram showing the important brainstem centers and subcortical and cortical structures. \cite{campbell, HanchemWongIbrahim2018}} 
    \end{minipage}
    \centering
    \label{fig:Vagus Afferent Network}
  \end{figure}

  The LC is characterized by widely diffused projections to both subcortical and cortical structures. The projections of the LC are small unmyelinated fibers, forming a wide antero-posterior branching network to reach the raphe nuclei, the cerebellum, and almost all areas of the midbrain and forebrain regions. The LC is the main source of norepinephrine (NE) in the brain \cite{AGHAJANIAN1977570}. NE is a neurotransmitter that has been associated with the clinical effects of VNS by preventing seizure development and by inducing long-term plastic changes that could restore a normal function of the brain circuitry. Indeed, short bursts of VNS increase neuronal firing in the LC, leading to elevations in NE concentrations. \cite{BergerVespa2021}

  Studies have demonstrated indirect projection of the LC to the DRN, which sends widespread projections to upper cortical regions. DRN appears to have a more delayed response to VNS \cite{HanchemWongIbrahim2018}.

  Vagal afferents project to the PBN by way of both the NTS and LC. Cell bodies within the PBN send diffuse outputs to forebrain structures including the thalamus, insular cortex, amygdala, and hypothalamus. Moreover, PBN likely plays an important role in regulating thalamocortical circuitry that may be implicated in seizure generation. Specifically, PBN activates the intralaminar nuclei of the thalamus, which in turn relays sensory signals to widespread cortical areas. \cite{HanchemWongIbrahim2018}

  \subsection*{The Vagus Afferent Network}
    \subsubsection*{Structural and Functional connectivity}
    Structural connectivity and functional connectivity are two concepts that describe different aspects of brain organization. Structural connectivity refers to the anatomical organization of the brain by means of fiber tracts that connect different brain regions. Functional connectivity refers to the statistical dependence or correlation of neural activity patterns between different brain regions. Structural connectivity is often measured by diffusion magnetic resonance imaging (dMRI). Functional connectivity is often measured by electro-encephalography (EEG) or functional magnetic resonance imaging (fMRI)\footnote{fMRI is a non invasive neuroimaging technique that detects the changes in blood oxygenation and flow that occur in response to neural activity}.

    The main difference between structural connectivity and functional connectivity is that structural connectivity reflects the physical architecture of the brain, while functional connectivity reflects the dynamic interactions of neural activity. Functional connectivity can emerge from direct or indirect structural connections, as well as from external inputs or intrinsic dynamics.

    \begin{figure}[h]
      \centering
      \includegraphics[width=0.8\textwidth]{images/structuralFunctional.png}
      \caption{Differences between structural and functional connectivity \cite{CABRAL201784}.}
      \label{fig:Structural and Functional Connections}
    \end{figure}

    \subsubsection*{Thalamocortical Connections}
    Thalamocortical connections are believed to be an important substrate of VNS responsiveness because they modulate cortical excitability, rendering the brain less susceptible to seizures. The thalamus receives direct inputs from the NTS and PBN \cite{BecksteadJoel2980}. More recently, thalamic activation measured by BOLD fMRI was associated with improved VNS treatment response in patients with seizures \cite{NarayananWatts2002}. The importance of thalamocortical connections in mediating the VNS antiseizure effect is further borne out by a study that utilized resting-state functional MRI (rs-fMRI) data pre-VNS implantation and found an association of greater VNS efficacy with larger connectivity between the thalami to the anterior cingulate cortex (ACC) and left insula \cite{IBRAHIM2017634}.
    % magari anche qua metti na foto carina se c'è
    Significantly greater FA was observed in VNS responders in posterior thalamic radiation lateralized to the left. % qua mettere una foto per fa capire cos'é il posterior thalamic radiation.
    Functional connectivity in MEG also supports the role of intrinsic thalamocortical connectivity in VNS responders, was found that a functional network is significantly more active in VNS responders. \cite{Mithani2019}

    % mettere una foto del thalamo

    \subsubsection*{Limbic Circuitry}
    The limbic system is a collection of neuronal structures involved in controlling emotion, memory, behavior, and motivation. The fornix is the main efferent tract of the hippocampus projecting to the mammillary bodies, nucleus accumbens, septal nuclei, anterior thalamic nuclei and cingulate cortex. While the stria terminalis forms the major input tract from the amygdala to the hypothalamus. 

    In fornix and stria terminalis, exclusively lateralized to the left, was observed a greater FA in VNS responders.
    A functional analysis performed by MEG\footnote{Magnetoencephalography: is a functional neuroimaging technique that maps brain activity by recording magnetic fields produced by electrical currents occurring naturally in the brain, using very sensitive magnetometers.} revealed a functional network with significantly greater connectivity in VN responders.
    These change in the structure and function connectivity imply that the limbic system is involve in an antiseizure action. \cite{Mithani2019, Mithani2020}

    % mettere una foto del limbic circuit fornix e hippocampus

\section{Review of DW-MRI in VNS}
  \subsection*{DTI}
  In a study of 56 children done by Mithani et al significantly greater FA (within the left size) was observed in VNS responders in the left internal capsule, external capsule, corona radiata, posterior thalamic radiation, fornix and stria terminalis, superior longitudinal fasciculus, inferior longitudinal fasciculus, and inferior front-occipital fasciculus. The mean FA value in these tracts was 0.352 (standard deviation SD = 0.048) in responders and 0.309 (SD = 0.064) in non responders. No significant voxels were observed in the right hemisphere. Furthermore, no statistically significant differences were observed in any other DTI parameters, including MD, radial diffusivity, and axial diffusivity. Healthy controls showed that the profile of responders was more closely related to healthy children than non responders. The mean FA value in significant tracts for matched controls was 0.377 (SD = 0.0274) in healthy controls. \cite{Mithani2019}.

  A study conducted on a 4-year-old boy with intractable epilepsy at 10 months after implantation of VNS showed increased FA in the right fimbria-fornix at the level of both cerebral peduncles. \cite{Fan2014}

